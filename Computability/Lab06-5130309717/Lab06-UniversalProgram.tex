\documentclass[12pt,a4paper]{article}

\usepackage{amsmath,amscd,amsbsy,amssymb,latexsym,url,bm,amsthm}
\usepackage{epsfig,graphicx,subfigure}
\usepackage{enumitem,balance}
\usepackage{wrapfig}
\usepackage{mathrsfs, euscript}
\usepackage[usenames]{xcolor}
\usepackage{hyperref}
\usepackage{multicol}

\theoremstyle{definition}

\newtheorem{theorem}{Theorem}
\newtheorem{lemma}[theorem]{Lemma}
\newtheorem{proposition}[theorem]{Proposition}
\newtheorem{corollary}[theorem]{Corollary}
\newtheorem{exercise}{Exercise}[section]
\newtheorem*{solution}{Solution}

\numberwithin{equation}{section}
\numberwithin{figure}{section}

\renewcommand{\thefootnote}{\fnsymbol{footnote}}

\newcommand{\postscript}[2]
 {\setlength{\epsfxsize}{#2\hsize}
  \centerline{\epsfbox{#1}}}

%changing 1.5 will give you different space between lines.
\renewcommand{\baselinestretch}{1.0}

\setlength{\oddsidemargin}{-0.365in}
\setlength{\evensidemargin}{-0.365in}
\setlength{\topmargin}{-0.3in}
\setlength{\headheight}{0in}
\setlength{\headsep}{0in}
\setlength{\textheight}{10.1in}
\setlength{\textwidth}{7in}


\begin{document}

\noindent\framebox[\linewidth]{\shortstack[c]{
\Large{\textbf{Lab06-Universal Program}}\vspace{1mm}\\
CS363-Computability Theory, Xiaofeng Gao, Spring 2016}}
\begin{center}
\footnotesize{\color{red}$*$ Please upload your assignment to FTP or submit a paper version on the next class}

\footnotesize{\color{red}$*$ If there is any problem, please contact: nongeek.zv@gmail.com }

\footnotesize{\color{blue}$*$ Name: Wenhao Zhu \quad StudentId: 5130309717 \quad Email: weehowe.z@gmail.com}
\end{center}


\begin{enumerate}%[topsep=0pt, partopsep=0pt, itemsep=2pt,parsep=2pt]

\item
\begin{enumerate}
\item Show that there is a decidable predicate $Q(x,y,z)$ such that
\begin{enumerate}
\item $y\in E_x$ if and only if $\exists z.Q(x,y,z)$
\item if $y\in E_x$ and $Q(x,y,z)$, then $\phi_x((z)_1)=y$.
\end{enumerate}
\begin{proof}
Note that $y \in E_x$ iff there exists some s such that $\phi_x(s) = y$, that is, iff the computation $P_x(s)$ halts with output y after a infinite some t steps. Thus under the effective coding $z = 2^s3^t$ we can define: $Q(x,y,z)=s_1(x,s,y,t)=s_1(x,(z)_1,y,(z)_2)$
Now Q is decidable by substitution.
\end{proof}
\item Deduce that there is a computable function $g(x,y)$ such that
\begin{enumerate}
\item $g(x,y)$ is defined if and only if $y\in E_x$.
\item if $y\in E_x$, then $g(x,y)\in W_x$ and $\phi_x(g(x,y))=y$; i.e. $g(x,y)\in \phi^{-1}_x(\{y\})$.
\end{enumerate}
\begin{proof}
Define $g(x,y) \simeq (\mu zQ(x,y,z))_1 $
\end{proof}
\item Deduce that if $f$ is a computable injective function (not necessarily total or surjective) then $f^{-1}$ is computable. (cf. exercise 2-5.4(1)).
\begin{proof}
$f$ is computable,accroding to unbounded minimization $\mu y(f(x)=y)$ is computable .Since $f^{-1}(y) = \mu y(f(x)=y)$ so $f^{-1}$is computable.
\end{proof}
\end{enumerate}
\item (cf. example 3-7.1(b)) Suppose that $f$ and $g$ are unary computable functions; assuming that $T_1$ has been formally proved to be decidable, prove formally that the function $h(x)$ defined by \\
    $h(x)=\left\{\begin{array}{ll}
    1& \mbox{if } x\in Dom(f) \mbox{ or } x\in Dom(g),\\
    \uparrow & \mbox{otherwise},
    \end{array}\right.$ is computable.
\begin{proof}
Because $f$ and $g$ are unary computable functions, there exists $e_1$, $e_2 \in N$ such that $f \simeq \phi_{e_1}$, $g \simeq \phi_{e_2}$. So, $h(x) \simeq sg(µt(T1(e_1, x, t)$ OR $T1(e_1, x, t)) + 1)$. $h(x)$ is defined if and only
if $\exists t, T1(e_1, x, t)$ or $T1(e_2, x, t)$. And when it is defined,it outputs 1.
$h(x)$ uses only one $µ$ operation and is composed of several primitive recursive functions and predicates, so, it is computable.
\end{proof}

\item Show that there is a total computable function $k(e_1, e_2)$ such that $\phi_{k(e_1,e_2)}(x)$ is the characteristic function for predicate ``$M_1(x)$ and $M_2(x)$", where $M_1$ and $M_2$ are both decidable predicate and $\phi_{e_1}=c_{M_1}$,  $\phi_{e_2}=c_{M_2}$.
\begin{proof}
Define $f(e_1,e_2,x) = \phi_{e_1}(x)\phi_{e_2}(x) = \psi_U(e_1, x)\psi_U(e_2,x)$. \\
We can know that $f(e_1, e_2, x)$ is also a characteristic function for ``$M_1(x)$ and $M_2(x)$". Because ``$M_1(x)$ and $M_2(x)$" are both decidable predicate, $f(e_1, e_2, x)$ must be a computable function. By s-m-n theorem,there is a total computable function $k(e_1, e_2)$ such that $\phi_{k(e_1,e_2)}(x) \simeq f(e_1, e_2, x)$.
\end{proof}

\item Show that there is a total computable function $s(x,y)$ such that for all $x,y$, $E_{s(x,y)}=W_x \cup E_y$.
\begin{proof}
For pair $x$,$y$,we can define such a function:
   $f(x,y,z)= \begin{cases} \frac{z}{2} & \mbox {z is even}\\
    \phi_x(\frac{z+1}{2}) & \mbox {z is odd}
        \end{cases}$ \\
By the s-n-m theory there exists a total computable function $s(x,y)$ such that for all $x$ and $y$, $\phi_{s(x,y)}(z) = f(x,y,z)$. So it is obvious that $E_s(x,y) = W_X\cup E_y$.
\end{proof}

\item Suppose that $f(x)$ is computable; show that there is a total computable function $k(x)$ such that for all $x$, $W_{k(x)}=f^{-1}(W_x)$.
\begin{proof}
Define     $g(x,y)=\left\{\begin{array}{ll}
    1& \mbox{if } f(y)\in W_x\\
    undefined & \mbox{otherwise}
    \end{array}\right.$ ,\\
As we know that $f(x)$ is computable, so $g(x,y)$ is also compuatble. By s-m-n theorem, there is a total computable function $k(x)$ such that $\phi_{k(x)}(y) \simeq g(x,y)$. And it is obvious that for any fixed x, the domain of $\phi_{k(x)}(y)$ is $f^{-1}(W_x)$ by definition of $g(x,y)$.
\end{proof}
\end{enumerate}



%========================================================================
\end{document}
