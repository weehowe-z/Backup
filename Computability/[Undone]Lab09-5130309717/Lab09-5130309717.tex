\documentclass[12pt,a4paper]{article}
\usepackage{amsmath,amsthm,amssymb,bm}
\usepackage[usenames]{xcolor}
\usepackage{hyperref}
\usepackage{graphicx}
\usepackage{epstopdf}
%\usepackage{float}


\newtheorem{theorem}{Theorem}
\newtheorem{lemma}[theorem]{Lemma}
\newtheorem{proposition}[theorem]{Proposition}
\newtheorem{corollary}[theorem]{Corollary}
\newtheorem{exercise}{Exercise}[section]
\newtheorem*{solution}{Solution}
\theoremstyle{definition}


\numberwithin{equation}{section}
\numberwithin{figure}{section}

\renewcommand{\thefootnote}{\fnsymbol{footnote}}

\newcommand{\postscript}[2]
 {\setlength{\epsfxsize}{#2\hsize}
  \centerline{\epsfbox{#1}}}

\renewcommand{\baselinestretch}{1.0}

\setlength{\oddsidemargin}{-0.365in}
\setlength{\evensidemargin}{-0.365in}
\setlength{\topmargin}{-0.3in}
\setlength{\headheight}{0in}
\setlength{\headsep}{0in}
\setlength{\textheight}{10.1in}
\setlength{\textwidth}{7in}
\makeatletter \renewenvironment{proof}[1][Proof] {\par\pushQED{\qed}\normalfont\topsep6\p@\@plus6\p@\relax\trivlist\item[\hskip\labelsep\bfseries#1\@addpunct{.}]\ignorespaces}{\popQED\endtrivlist\@endpefalse} \makeatother
\makeatletter
\renewenvironment{solution}[1][Solution] {\par\pushQED{\qed}\normalfont\topsep6\p@\@plus6\p@\relax\trivlist\item[\hskip\labelsep\bfseries#1\@addpunct{.}]\ignorespaces}{\popQED\endtrivlist\@endpefalse} \makeatother



\begin{document}
\noindent

%========================================================================
\noindent\framebox[\linewidth]{\shortstack[c]{
\Large{\textbf{Lab09-Recursively Enumerable Set(2)}}\vspace{1mm}\\
CS363-Computability Theory, Xiaofeng Gao, Spring 2016}}
\begin{center}
\footnotesize{\color{red}$*$ Please upload your assignment to FTP or submit a paper version on the next class}

\footnotesize{\color{red}$*$ If there is any problem, please contact: steinsgate@sjtu.edu.cn}

\footnotesize{\color{blue}$*$ Name:Wenhao Zhu    \quad StudentId: 5130309717 \quad Email: weehowe.z@gmail.com}
\end{center}

\begin{enumerate}
\item Suppose $A$ is an r.e.~set. Prove the following statements.
\begin{enumerate}
\item Show that the sets $\bigcup\limits_{x\in A}W_x$ and $\bigcup\limits_{x\in A}E_x$ are both r.e.
\begin{proof}
    As $x \in W_x$is partially decidable.So there $\exists x \in A$ such that $z \in W_x$
    is partially decidable.$\exists x \in A$ that $z \in W_x$ is equal to $x \in  \bigcup\limits_{x\in A}W_x$
    ,so it is also partially decidable.Then the set$\bigcup\limits_{x\in A}W_x$ is r.e. The same as $\bigcup\limits_{x\in A}E_x$, we can prove in the same way.
\end{proof}
\item Show that $\bigcap\limits_{x\in A}W_x$ is not necessarily r.e. (\emph{Hint}: $\forall t \in \mathbb{N}$ let $K_t=\{x:P_x(x)\downarrow \mbox{ in t steps}\}$. Show that for any $t$, $K_t$ is recursive; moreover $K=\bigcup\limits_{t\in\mathbb{N}}K_t$ and $\overline{K}=\bigcup\limits_{t\in\mathbb{N}}\overline{K}_t$.)
\begin{proof}
    As we konw $z \in \bigcap\limits_{x\in A}W_x$ means $\forall x \in A $, $z \in W_x$, $z \in W_x$ is partially decidable but $\forall x \in A $, however $z \in W_x$ is partially decidable, so the set is not necessarily decidable.
\end{proof}
\end{enumerate}

\item Prove that $A\subseteq\mathbb{N}^n$ is r.e.~iff $A=\varnothing$ or there is a total computable function $f:\mathbb{N}\rightarrow\mathbb{N}^n$ such that $A=Ran(\bm{f})$. (A \emph{computable function} $\bm{f}$ from $\mathbb{N}$ to $\mathbb{N}^n$ is an $n$-tuple $\bm{f}=(f_1,\ldots,f_n)$ where each $f_i$ is a unary computable function and $\bm{f}(x)=(f_1(x),\ldots,f_n(x))$.)
\begin{proof}
    if $A=\varnothing$ then $x \in A$ is p.d. so A is r.e. If there is a total computable function $f:\mathbb{N}\rightarrow\mathbb{N}^n$ such that $A=Ran(\bm{f})$,then $A=(f_1(x),\ldots,f_n(x))$.So N is m-reducable to A. Since N is r.e.,A is also r.e.
\end{proof}

\item Suppose that $f$ is a total computable function, $A$ is a recursive set and $B$ is an r.e.set. Show that $f^{-1}(A)$ is recursive and that $f(A)$, $f(B)$ and $f^{-1}(B)$ are r.e. but not necessarily recursive. What extra information about these sets can be obtained if $f$ is a bijection?
\begin{proof}
    $x \in f^{-1}(A)$ is equal to $f(x) \in A$ and since A is recursive, then $f(x) \in A$  is decidable so $x \in f^{-1}(A)$ is decidable so $f^{-1}(A)$ is recursive.$x \in f(A)$ is equal to $\exists y \in N,y \in A$ and $f(y)=x$
    and since $y \in A$ and $f(y)=x$ is decidable,it is partially decidable so $f(A)$ is r.e.$x \in f(B)$ is equal to $\exists y \in N,y \in B$ and $f(y)=x$
    ,it is p.d. so $f(B)$ is r.e. Similarly we can prove $f^{-1}(B)$ is r.e. As $f$ is a total computable function, if $f$ is bijective, then $f(A)$ is recursive.
\end{proof}

\item A set $D$ is the difference of r.e. sets (\emph{d.r.e.}) iff $D=A-B$ where $A,B$ are both \emph{r.e.}.
\begin{enumerate}
\item Show that the set of all \emph{d.r.e.} sets is closed under the formation of intersection.
\begin{proof}
    For $\forall d.r.e$ set $M=A-B$ and $N=C-D$ where A,B,C,D are all r.e. $M \bigcap N = (A-B) \bigcap (C-D) = A\bigcap C + B \bigcap D - A \bigcap D - B \bigcap C$.so it is also $d.r.e$. Thus $d.r.e $ sets is closed under the formation of intersection.
\end{proof}
\item Show that if $C_n = \{x \mid |W_x|=n \}$, then $C_n$ is \emph{d.r.e.} for all $n \ge 0$.
\begin{proof}
    We can define a program that for numbers in N, if number $\in W_x$ more than n, then it will halt, otherwise we let it undefined. Thus it is d.r.e
\end{proof}

\end{enumerate}

\end{enumerate}

\end{document}