\documentclass[12pt,a4paper]{article}

\usepackage{amsmath,amscd,amsbsy,amssymb,latexsym,url,bm,amsthm}
\usepackage{epsfig,graphicx,subfigure}
\usepackage{enumitem,balance}
\usepackage{wrapfig}
\usepackage{mathrsfs, euscript}
\usepackage[usenames]{xcolor}
\usepackage{hyperref}
\usepackage{multicol}

\theoremstyle{definition}

\newtheorem{theorem}{Theorem}
\newtheorem{lemma}[theorem]{Lemma}
\newtheorem{proposition}[theorem]{Proposition}
\newtheorem{corollary}[theorem]{Corollary}
\newtheorem{exercise}{Exercise}[section]
\newtheorem*{solution}{Solution}

\numberwithin{equation}{section}
\numberwithin{figure}{section}

\renewcommand{\thefootnote}{\fnsymbol{footnote}}

\newcommand{\postscript}[2]
 {\setlength{\epsfxsize}{#2\hsize}
  \centerline{\epsfbox{#1}}}

%changing 1.5 will give you different space between lines.
\renewcommand{\baselinestretch}{1.0}

\setlength{\oddsidemargin}{-0.365in}
\setlength{\evensidemargin}{-0.365in}
\setlength{\topmargin}{-0.3in}
\setlength{\headheight}{0in}
\setlength{\headsep}{0in}
\setlength{\textheight}{10.1in}
\setlength{\textwidth}{7in}


\begin{document}

\noindent\framebox[\linewidth]{\shortstack[c]{
\Large{\textbf{Lab04-Church's Thesis}}\vspace{1mm}\\
CS363-Computability Theory, Xiaofeng Gao, Spring 2016}}
\begin{center}
\footnotesize{\color{red}$*$ Please upload your assignment to FTP or submit a paper version on the next class}

\footnotesize{\color{red}$*$ If there is any problem, please contact: nongeek.zv@gmail.com }

\footnotesize{\color{blue}$*$ Name:Wenhao Zhu \quad StudentId: 5130309717 \quad Email: weehowe.z@gmail.com}
\end{center}


\begin{enumerate}%[topsep=0pt, partopsep=0pt, itemsep=2pt,parsep=2pt]

  \item Suggest the natural definition of computability on domain $\mathbb{Q}$ (rational numbers).
  
    \begin{solution}
        Definition:
        \begin{itemize}
            \item $Z(q) = 0$
            
            \item $q=\left\{\begin{array}{ll}
            \frac{m}{n} & \mbox{ if } q > 0 \mbox{ and }  gcd(m,n)=1\\
       -\frac{m}{n} & \mbox{ if } q < 0 \mbox{ and }  gcd(m,n)=1
        \end{array}\right.$
        
            \item $t = \mu x(gcd(\pi_1(\pi(m,n))))$
            
            \item $U_i^n (q_1,q_2, \dots, q_n)= q_i$
        \end{itemize}
      \end{solution}

 \item Define $f(n)$ as the $n$-th digit in the decimal expansion of $e$. Use Church's Thesis to prove that $f$ is computable. ($e$ is the the base of the natural logarithm and can be calculated as the sum of the infinite series: $e = \sum\limits_{n=0}^{\infty}\dfrac{1}{n!}$)

\begin{proof}
As we can extend the $e$ as the sum of the infinite series. Let $S_k = \sum\limits_{n=0}^{\infty}\dfrac{1}{n!}$, by theory of infinite series, $S_k < e < S_k + \dfrac{1}{k!} $. Since $S_k$ is rational, the decimal expansion of $S_k$ can be effectively
calculated to any desired number of places using long division. Thus the effective method for calculating f(n) (given a number n) can be described as:\\
Find the first $N \le n + 1$ such that the decimal expansion
$S_N = a_0 \cdot a_1 \cdot a_2 \cdot \cdot \cdot a_na_{n+1} \cdot \cdot \cdot a_N \cdot \cdot \cdot $ does not have all of $a_{n+1} · · · a_N$
equal to 9. Then put $f(n) = a_n$. \\
To see that this gives the required value, suppose that $a_m \neq 9$ with
$n < m \le N$. Hence $a_0 \cdot a_1 \cdot \cdot \cdot a_n \cdot \cdot \cdot a_m \cdot \cdot \cdot < e < a_0 \cdot a_1 \cdot \cdot \cdot a_n \cdot \cdot \cdot (a_m + 1) \cdot \cdot \cdot $. So
the $n$-th decimal place of $e$ is indeed $a_n$. \\
Thus by Church's Thesis, $f$ is computable.
\end{proof}

\item Suppose there is a two-tape Turing Machine $M$ with alphabet $\Gamma = \{ \triangleright, \triangleleft, \Box, 1 \}$ and state set $Q = \{ q_s, q_1, q_2, q_h \}$. $M$ has the following specifications. Transform $M$ into a single-tape Turing Machine $\widetilde{M}$, and write down the new alphabet and specifications.
\begin{eqnarray*}
   \langle q_s,\triangleright,\triangleright  \rangle & \rightarrow & \langle q_1, \triangleright, S, R\rangle \\
   \langle q_1,\triangleright,\Box \rangle & \rightarrow & \langle q_2, 1, R, R\rangle \\
   \langle q_2, 1,\Box \rangle & \rightarrow & \langle q_2, 1, R, R\rangle \\
   \langle q_2, \triangleleft,\Box \rangle & \rightarrow & \langle q_h, \triangleleft, S, S\rangle
\end{eqnarray*}
\begin{solution}
  Alphabet $\Gamma = \{ \triangleright, \triangleleft, \Box, 1 \}$, and the specifications are:
\begin{eqnarray*}
   \langle q_s,\triangleright \rangle & \rightarrow & \langle q_1, \triangleright, R\rangle \\
   \langle q_1, 1 \rangle & \rightarrow & \langle q_1, 1, R\rangle \\
   \langle q_1, \triangleleft \rangle & \rightarrow & \langle q_2, 1, R \rangle \\
   \langle q_2, \Box \rangle & \rightarrow & \langle q_h, \triangleleft, R \rangle   
\end{eqnarray*}
\end{solution}

\item Design a three-tape TM $M$ that computes the function $f(x,y) = x \% y$, where both $m$ and $n$ belong to the natural number set $\mathbb{N}$. The alphabet is $\{1, \Box, \triangleright, \triangleleft\}$, where the input on the first tape is $x+1$ ``1'''s and $y+1$ ``1'''s with a ``$\Box$'' as the separation. Below is the initial configurations for input $(x,y)$. The result is the number of ``1'''s on the output tape with the pattern of $\triangleright 111
\cdots 111\triangleleft$. First describe your design and then write the specifications of $M$ in the form like $\langle q_S, \triangleright, \triangleright, \triangleright \rangle \rightarrow \langle q_1, \triangleright, \triangleright, R, R, R\rangle$ and explain the transition functions in detail (especially the meaning of each state).

  \begin{tabular}{ll|c|c|c|c|c|c|c|c|c|c|c|c|c|c}
      & \multicolumn{14}{c}{Initial Configurations}\\[5pt]
      \cline{2-16}
      Tape 1:& & $\triangleright$ &  1  & 1 & $\cdots$ & 1 & 1 & $\Box$ & 1 & 1 & $\cdots$ & 1 & 1 & $ \triangleleft$ & \\
      \cline{2-16}
      \multicolumn{2}{c}{} & \multicolumn{1}{c}{$\uparrow$} & \multicolumn{5}{c}{\color{blue}$\leftarrow x+1 \mbox{ squares}\rightarrow$} & \multicolumn{1}{c}{} & \multicolumn{5}{c}{\color{blue}$\leftarrow y+1 \mbox{ squares}\rightarrow$} & \multicolumn{2}{c}{}\\[4pt]
      \cline{2-16}
      Tape 2:& & $\triangleright$ & $\Box$ & $\Box$ & \multicolumn{7}{c|}{$\cdots$ \quad $\cdots$ \quad $\cdots$} & $\Box$ & $\Box$ & $\Box$ &\\
      \cline{2-16}
      \multicolumn{2}{c}{} & \multicolumn{1}{c}{$\uparrow$} & \multicolumn{11}{c}{}\\[4pt]
      \cline{2-16}
      Tape 3:& & $\triangleright$ & $\Box$ & $\Box$ & \multicolumn{7}{c|}{$\cdots$ \quad $\cdots$ \quad $\cdots$} & $\Box$ & $\Box$ & $\Box$ &\\
      \cline{2-16}
      \multicolumn{2}{c}{} & \multicolumn{1}{c}{$\uparrow$} & \multicolumn{13}{c}{}\\
      \end{tabular}

    \begin{solution}
    Begin.
    \begin{eqnarray*}
   \langle q_s,\triangleright,\triangleright,  \triangleright  \rangle & \rightarrow & \langle q_1, \triangleright,\triangleright, R, R, S   \rangle \\
\end{eqnarray*}
    Copy $x+1$ '1' to Tape 2
\begin{eqnarray*}
   \langle q_1,1,\Box,  \Box  \rangle & \rightarrow &
   \langle q_1, 1,\Box, R, R, S   \rangle \\
   \langle q_1,\Box ,\Box,  \Box  \rangle & \rightarrow &
   \langle q_2, \triangleleft,\Box, R, S, S   \rangle \\
\end{eqnarray*}
    Move pointer of Tape 2 to the end.
    \begin{eqnarray*}
   \langle q_2,1,\triangleright,  \Box  \rangle & \rightarrow &
    \langle q_2, \triangleright,\Box, R, S, S   \rangle \\
    \langle q_2,\triangleright,\triangleright,  \Box  \rangle & \rightarrow &
    \langle q_3, \triangleright,\Box, L, L, S   \rangle \\
    \end{eqnarray*}
    Move back pointer of Tape 2 and Tape 3 at the same time and get the result
     \begin{eqnarray*}
   \langle q_3,1,1,  \Box  \rangle & \rightarrow &
    \langle q_3, 1,\Box, L, L, S   \rangle \\
    \end{eqnarray*}
    If $x=y$ or $x>y$, calculate $x = x - y$.
     \begin{eqnarray*}
   \langle q_3,\Box,\triangleright,  \Box  \rangle & \rightarrow &
    \langle q_4, \triangleright,\Box, R, R, S   \rangle \\
   \langle q_3,\Box,1,  \Box  \rangle & \rightarrow &
    \langle q_4, \triangleright,\Box, R, R, S   \rangle \\
   \langle q_3,1,\triangleright,  \Box  \rangle & \rightarrow &
    \langle q_7, \triangleright,\Box, S, R, R   \rangle \\
    \end{eqnarray*}
    Get $x = x - y$. 
    \begin{eqnarray*}
   \langle q_4,1,1,  \Box  \rangle & \rightarrow &
    \langle q_4, 1,\Box, R, R, S   \rangle \\
    \langle q_4,\triangleleft,\triangleleft,  \Box  \rangle & \rightarrow &
    \langle q_4, \triangleleft,\Box, L, S, S   \rangle \\
     \langle q_4,1,\triangleleft,  \Box  \rangle & \rightarrow &
    \langle q_5, \Box,\Box, L, L, S   \rangle \\
     \langle q_5,1,1,  \Box  \rangle & \rightarrow &
    \langle q_5, \Box,\Box, L, L, S   \rangle \\
    \langle q_5, \Box ,1,  \Box  \rangle & \rightarrow &
    \langle q_6, \Box,\Box, L, L, S   \rangle \\
    \langle q_6, \Box ,1,  \Box  \rangle & \rightarrow &
    \langle q_2, \triangleleft,\Box, R, S, S   \rangle \\
    \end{eqnarray*}
    Else perform copy from Tape 2 to Tape 3 and halt.
     \begin{eqnarray*}
   \langle q_7,\Box ,1,  \Box  \rangle & \rightarrow &
    \langle q_7, \Box,1, S, R, R   \rangle \\
     \langle q_7,\Box ,\triangleleft,  \Box  \rangle & \rightarrow &
    \langle q_H, \triangleleft,\triangleleft, S, S, S   \rangle \\
    \end{eqnarray*}

    \end{solution}

\end{enumerate}


%========================================================================
\end{document}