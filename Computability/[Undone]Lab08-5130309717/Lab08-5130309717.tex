\documentclass[12pt,a4paper]{article}
\usepackage{amsmath,amsthm,amssymb}
\usepackage[usenames]{xcolor}
\usepackage{hyperref}
\usepackage{graphicx}
\usepackage{epstopdf}
%\usepackage{float}


\newtheorem{theorem}{Theorem}
\newtheorem{lemma}[theorem]{Lemma}
\newtheorem{proposition}[theorem]{Proposition}
\newtheorem{corollary}[theorem]{Corollary}
\newtheorem{exercise}{Exercise}[section]
\newtheorem*{solution}{Solution}
\theoremstyle{definition}


\numberwithin{equation}{section}
\numberwithin{figure}{section}

\renewcommand{\thefootnote}{\fnsymbol{footnote}}

\newcommand{\postscript}[2]
 {\setlength{\epsfxsize}{#2\hsize}
  \centerline{\epsfbox{#1}}}

\renewcommand{\baselinestretch}{1.0}

\setlength{\oddsidemargin}{-0.365in}
\setlength{\evensidemargin}{-0.365in}
\setlength{\topmargin}{-0.3in}
\setlength{\headheight}{0in}
\setlength{\headsep}{0in}
\setlength{\textheight}{10.1in}
\setlength{\textwidth}{7in}
\makeatletter \renewenvironment{proof}[1][Proof] {\par\pushQED{\qed}\normalfont\topsep6\p@\@plus6\p@\relax\trivlist\item[\hskip\labelsep\bfseries#1\@addpunct{.}]\ignorespaces}{\popQED\endtrivlist\@endpefalse} \makeatother
\makeatletter
\renewenvironment{solution}[1][Solution] {\par\pushQED{\qed}\normalfont\topsep6\p@\@plus6\p@\relax\trivlist\item[\hskip\labelsep\bfseries#1\@addpunct{.}]\ignorespaces}{\popQED\endtrivlist\@endpefalse} \makeatother



\begin{document}
\noindent

%========================================================================
\noindent\framebox[\linewidth]{\shortstack[c]{
\Large{\textbf{Lab08-Recursively Enumerable Set}}\vspace{1mm}\\
CS363-Computability Theory, Xiaofeng Gao, Spring 2016}}
\begin{center}
\footnotesize{\color{red}$*$ Please upload your assignment to FTP or submit a paper version on the next class}

\footnotesize{\color{red}$*$ If there is any problem, please contact: steinsgate@sjtu.edu.cn}

\footnotesize{\color{blue}$*$ Name: Wenhao Zhu \quad StudentId: 5130309717 \quad Email: weehowe.z@gmail.com}
\end{center}

\begin{enumerate}
\item Let $A,B$ be subsets of $\mathbb{N}$. Define sets $A\oplus B$ and $A\otimes B$ by
$$
\begin{array}{l}
A\oplus B=\{2x \mid x\in A\}\cup\{2x+1 \mid x\in B\},\\[3pt]
A\otimes B=\{\pi (x,y) \mid x\in A \mbox{ and } y\in B\},
\end{array}$$
where $\pi$ is the pairing function $\pi(x,y)=2^x(2y+1)-1$. Prove that
  \begin{enumerate}
  \item $A\oplus B$ is recursive iff $A$ and $B$ are both recursive.
  \item If $A,B\neq\varnothing$, then $A\otimes B$ is recursive iff $A$ and $B$ are both recursive.
  \end{enumerate}

  \begin{solution}
  $ $
  \begin{enumerate}
  \item (if) $c_{A\oplus B}(x)=\left\{\begin{array}{ll}
    1, & \mbox{if } x \mbox{ is even and } x/2\in A, \mbox{or if } x \mbox{ is odd and } (x-1)/2\in B,\\
    0, & \mbox{otherwise.}
    \end{array}\right.$
    Since $A$ and $B$ are both recursive, $c_{A\oplus B}(x)$ is computable, so $A\oplus B$ is recursive.

    (only if) $c_{A}(x)=\left\{\begin{array}{ll}
    1, & \mbox{if } 2x \in A\oplus B,\\
    0, & \mbox{otherwise.}
    \end{array}\right.$
    Since $A\oplus B$ is recursive, $c_{A}(x)$ is computable, so $A$ is recursive. Similarly, $B$ is recursive.
  \item (if) $c_{A\otimes B}(x)=\left\{\begin{array}{ll}
    1, & \mbox{if } \pi_1(x)\in A \mbox{ and } \pi_2(x)\in B,\\
    0, & \mbox{otherwise.}
    \end{array}\right.$
    Since $A$ and $B$ are both recursive, $c_{A\otimes B}(x)$ is computable, so $A\otimes B$ is recursive.

    (only if) Since $A,B\ne \emptyset$, we can find an element $a$ in $A$ and $b$ in $B$. Then $c_{A}(x)=c_{A\otimes B}(\pi(x,b))$, $c_{B}(x)=c_{A\otimes B}(\pi(a,x))$. By substitution, $c_{A}(x)$ and $c_{B}(x)$ are computable, so $A$ and $B$ are recursive.
  \end{enumerate}
  \end{solution}

\item Which of the following sets are recursive? Which are r.e.? Which have r.e.~complement? Prove your judgements.
  \begin{enumerate}
  \item $\{x \mid P_m(x) \downarrow \mbox{ in } t \mbox{ or fewer steps }\}$ ($m,t$ are fixed).
  \item $\{x \mid x \mbox{ is a power of 2}\}$;
  \item $\{x \mid \phi_x \mbox{ is injective}\}$;
  \item $\{x \mid y\in E_x\}$ ($y$ is fixed);
  \end{enumerate}

  \begin{solution}
  $ $
  \begin{enumerate}
  \item  $c_{A}(x)=\left\{\begin{array}{ll}
         1, & \mbox{if } \textsf{j}(m,x,t)=0,\\
         0, & \mbox{if } \textsf{j}(m,x,t)\neq0.
      \end{array}\right.$
      $\textsf{j}(m,x,t)$ is primitive recursive, so $c_{A}(x)$ is computable, so $A$ is recursive. So it is r.e. and has r.e.~complement.
  \item  $c_{A}(x)=\left\{\begin{array}{ll}
         1, & \mbox{if } 2^{(x)_1}=x,\\
         0, & \mbox{otherwise.}
      \end{array}\right.$
      $c_{A}(x)$ is computable, so $A$ is recursive. So it is r.e. and has r.e.~complement.
  \item Define $f(x,y)=\left\{\begin{array}{ll}
         y, & \mbox{if }x\in W_x,\\
         \uparrow, & \mbox{otherwise}.
      \end{array}\right.$ $f(x,y)$ is computable. According to s-m-n theorem there exists a total computable function $k$ such that $\phi_{k(x)}(y)\simeq f(x,y)$. Clearly, $k:K\leq_m \{x \mid \phi_x \mbox{ is injective}\}$, so it is not recursive.\\
      It has r.e. compliment because $\phi_x$ is not injective $\Leftrightarrow$ $\exists a\exists b.(\phi_x(a)= \phi_x(b) \wedge a\neq b)$. The right side is a partial decidable predicate thus $\{x \mid \phi_x \mbox{ is not injective}\}$ is r.e.\\
      $\{x \mid \phi_x \mbox{ is injective}\}$ is not r.e. because if it is r.e., $\{x \mid \phi_x \mbox{ is injective}\}$ would be recursive.
  \item Define $f(x,z)=\left\{\begin{array}{ll}
         y, & \mbox{if }x\in W_x,\\
         \uparrow, & \mbox{otherwise}.
      \end{array}\right.$ $f(x,z)$ is computable. According to s-m-n theorem there is a total computable function $k$ such that $\phi_{k(x)}(z)=f(x,z)$. Clearly, $k:\{x\mid x\in W_x\}\leq_m \{x \mid y\in E_x\}$, so it is not recursive.\\
      It is r.e. because $y\in E_x\Leftrightarrow\exists a\exists t(P_x(a)\downarrow y \mbox{ in } t \mbox{ steps})$. The right side is partially decidable thus $\{x \mid y\in E_x\}$ is r.e.\\
      It doesn't have r.e. compliment because if so, $\{x \mid y\in E_x\}$ would be recursive.
  \end{enumerate}
  \end{solution}

\item Prove following statements.
\begin{enumerate}
\item Let $B\subseteq\mathbb{N}$ and $n>1$; prove that $B$ is r.e. then the predicate $M(x_1,\ldots,x_n)$ given by ``$M(x_1,\ldots,x_n)\equiv2^{x_1}3^{x_2}\ldots p_n^{x_n}\in B$" is partially decidable.
\item Prove that $A\subseteq\mathbb{N}^n$ is r.e. iff $\{2^{x_1}3^{x_2}\ldots p_n^{x_n} \mid (x_1,\ldots,x_n)\in A\}$ is r.e..
\end{enumerate}
\begin{solution}
$ $
\begin{enumerate}
\item As we know that $B$ is r.e., so $\chi_B(x)$ is computable. Define $f(x_1,x_2,\ldots,x_n)=2^{x_1}3^{x_2}\ldots p_n^{x_n}$, then \\ $\chi_M(x_1,x_2,\ldots,x_n)=\chi_B(f(x_1,x_2,\ldots,x_n))$. By substitution, $\chi_M$ is computable, so $M$ is partially decidable.
\item 
\end{enumerate}
\end{solution}
\end{enumerate}

\end{document}
