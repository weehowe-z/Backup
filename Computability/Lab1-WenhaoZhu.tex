\documentclass[12pt,a4paper]{article}
\usepackage{ctex}
\usepackage{amsmath,amscd,amsbsy,amssymb,latexsym,url,bm,amsthm}
\usepackage{epsfig,graphicx,subfigure}
\usepackage{enumitem,balance}
\usepackage{wrapfig}
\usepackage{mathrsfs, euscript}
\usepackage[usenames]{xcolor}
\usepackage{hyperref}
\usepackage[boxed]{algorithm2e}

\newtheorem{theorem}{Theorem}[section]
\newtheorem{lemma}[theorem]{Lemma}
\newtheorem{proposition}[theorem]{Proposition}
\newtheorem{corollary}[theorem]{Corollary}
\newtheorem{exercise}{Exercise}[section]
\newtheorem*{solution}{Solution}
\theoremstyle{definition}


\numberwithin{equation}{section}
\numberwithin{figure}{section}

\renewcommand{\thefootnote}{\fnsymbol{footnote}}

\newcommand{\postscript}[2]
 {\setlength{\epsfxsize}{#2\hsize}
  \centerline{\epsfbox{#1}}}

\renewcommand{\baselinestretch}{1.0}

\setlength{\oddsidemargin}{-0.365in}
\setlength{\evensidemargin}{-0.365in}
\setlength{\topmargin}{-0.3in}
\setlength{\headheight}{0in}
\setlength{\headsep}{0in}
\setlength{\textheight}{10.1in}
\setlength{\textwidth}{7in}
\makeatletter \renewenvironment{proof}[1][Proof] {\par\pushQED{\qed}\normalfont\topsep6\p@\@plus6\p@\relax\trivlist\item[\hskip\labelsep\bfseries#1\@addpunct{.}]\ignorespaces}{\popQED\endtrivlist\@endpefalse} \makeatother
\makeatletter
\renewenvironment{solution}[1][Solution] {\par\pushQED{\qed}\normalfont\topsep6\p@\@plus6\p@\relax\trivlist\item[\hskip\labelsep\bfseries#1\@addpunct{.}]\ignorespaces}{\popQED\endtrivlist\@endpefalse} \makeatother



\begin{document}
\noindent

%========================================================================
\noindent\framebox[\linewidth]{\shortstack[c]{
\Large{\textbf{Lab01-Proof}}\vspace{1mm}\\
CS363-Computability Theory, Xiaofeng Gao, Spring 2016}}
\begin{center}
\footnotesize{\color{red}$*$ Please upload your assignment to TA's FTP. Contact nongeek.zv@gmail.com for any questions.}

\footnotesize{\color{blue}$*$ Name:Wenhao Zhu \quad StudentId: 5130309717 \quad Email: weehowe.z@gmail.com}
\end{center}

\begin{enumerate}

\item Prove that for any integer $n>2$, there is a prime $p$ satisfying $n<p<n!$. {\color{blue}(Hint: consider a prime factor $p$ of $n!-1$ and use proof by contradiction)}

\begin{proof} Assume that for any integer $n>2$, there is no prime $p$ satisfying $n<p<n!$. That means for any $x$ satisfying $n<x<n!$ is not a prime.Then we consider a prime factor $p$ of $n!-1$, if $p<=n$, then p is also a prime factor of $n!$ according to the definition of factorial. Since $p|n!$ and $p|n!-1$ , we know $p|1$, there is a contradiction. So $p$ must be $p>n$, then we find a prime $p$ satisfying $n<p<n!$ which contradicts the assumption. Therefore, for any integer $n>2$, there is a prime $p$ satisfying $n<p<n!$.
\end{proof}

\item Use minimal counterexample principle to prove that: for every integer $n>17$, there exist integers $i_n\ge 0$ and $j_n\ge 0$, such that $n = i_n \times 4 + j_n \times 7$.

\begin{proof} If $n = i_n \times 4 + j_n \times 7$ is not true for every integer $n>17$, then there are values of n for which makes the equation false, and there must be a smallest such value, say $n = k$. Since $18=1 \times 4 + 2 \times 7$, $19=3 \times 4 + 1 \times 7$, $20=5 \times 4 + 0 \times 7$. we have $k \ge 21$.  Since $k$ is the smallest value, we know $k-1$ satisfying the equation, which means there exist integers $i_{k-1}\ge 0$ and $j_{k-1}\ge 0$, such that $k-1 = i_{k-1} \times 4 + j_{k-1} \times 7$. However, we have:

\begin{gather*}
k = i_{k-1} \times 4 + j_{k-1} \times 7 + 1 \\ 
\Leftrightarrow \quad k = (i_{k-1}+2) \times 4 + (j_{k-1}-1) \times 7\\
\Leftrightarrow \quad k = (i_{k-1}-5) \times 4 + (j_{k-1}+1) \times 7
\end{gather*}

(1) If $j_{k-1} \ge 1$, denote $i_k = i_{k-1}+2$, $j_k = j_{k-1}-1$, then we know $i_k > i_{k-1} \ge 0$, $j_k = j_{k-1}-1 \ge 0$. Thus $n = i_n \times 4 + j_n \times 7$ is still true, we have derived a contradiction. \\
(2) If $j_{k-1}-1 = 0$, then we know $i_{k-1} = (k-1) / 4 \ge 5$, denote $i_k = i_{k-1}-5$, $j_k = j_{k-1}+1$. We can get $i_k = i_{k-1}-5 \ge 0$ and $j_k = j_{k-1}+1\ge 0$. Thus $n = i_n \times 4 + j_n \times 7$ is still true, we have derived a contradiction. \\
In conclusion, for every integer $n>17$, there exist integers $i_n\ge 0$ and $j_n\ge 0$, such that $n = i_n \times 4 + j_n \times 7$.

\end{proof}

\item Suppose $a_0=1$, $a_1=2$, $a_2=3$, $a_k=a_{k-1}+a_{k-2}+a_{k-3}$ for $k \ge 3$. Use strong principle of mathematical induction to prove that $a_n \le 2^n$ for all integers $n\ge 0$.

\begin{proof} Define $P(n)$ be the statement that $a_n \le 2^n$. We will try to prove that $P(n)$ is true for every integer $n \ge 0$. \\
\textbf{Basis step.} $P(0) = 1 \le 2^0$, $P(1) = 2 \le 2^1$, $P(2) = 3 \le 2^2$, thus $P(0)$, $P(1)$, $P(2)$ are true. \\
\textbf{Induction hypothesis.} For $k \ge 0$ and $0 \le n \le k$, $P(n)$ is true. (Strong Principle) \\
\textbf{Proof of induction step.} Then we prove $P(k+1)$. \\

\begin{align*}
a_{k+1} =a_{k}+a_{k-1}+a_{k-2} \\
\le 2^k+2^{k-1}+2^{k-2} \\
\le 7 \times 2^{k-2} \\
\le 2^{k+1}
\end{align*}

$P(k+1)$ is true, thus $a_n \le 2^n$ for all integers $n\ge 0$.
\end{proof}

\item Consider the following loop, written in pseudocode:

\begin{center}
\begin{minipage}[b]{0.2\textwidth}
\begin{algorithm}[H]
  \While{B}{
   $S$\;
  }
\end{algorithm}
\end{minipage}
\end{center}


A condition $P$ is called an invariant of the loop if whenever $P$ and $B$ are both true, and $S$ is executed once, $P$ is still true.

\begin{enumerate}
  \item Prove that if $P$ is an invariant of the loop, and $P$ is true before the first iteration of the loop, then if the loop eventually terminates (i.e., after some number of iterations, $B$ is false), $P$ is still true.
  \begin{proof} Suppose the loop totally runs $n$ iterations, thus $B$ is true in the first $n$ iterations and change to false in the $n+1$ iteration. As $P$ is an invariant of the loop, and it is true before the loop, thus $P$ and $B$ are both true at the beginning, therefore, after the iteration, $P$ is still true. This situation remains for n iterations until $B$ change to false. At the end of this iteration, $P$ is true. Then in the next iteration, $S$ won't be excuted, nothing has changed, so $P$ remains to be true.  
  \end{proof}
  
  \item Suppose $x$ and $y$ are integer variables, and initally $x\ge 0$ and $y > 0$. Consider the following program fragment:

\begin{center}
\begin{minipage}[b]{0.25\textwidth}
\begin{algorithm}[H]
   $q$ = 0\;
   $r$ = $x$\;
   \While{$r \ge y$}{
      $q = q+1$\;
      $r = r-y$\;
   }
\end{algorithm}
\end{minipage}
\end{center}

By considering the condition ($r\ge 0) \wedge (x=q \times y+r$), prove that when this loop terminates, the values of $q$ and $r$ will be the integer quotient and remainder, respectively, when $x$ is divided by $y$; in other words, $x=q \times y+r$ and $0 \le r <y$.

  \begin{proof} Define $P$ is $x=q \times y+r$ and $0 \le r <y$. Then we prove $P$ is invariant. \\
  Before the iteration, as $r=x$, $x=0 \times y + x = q \times y+r$, $P$ is true.\\
  Whenever $P$ and {$r \ge y$} are true, during the iteration, $q^{'} = q+1$, $r^{'}=r-y$, $x^{'}=x$, $y^{'}=y$. $q^{'}$, $r^{'}$ are the value of $p$, $q$ after the iteration. We know: $x^{'} = x = q \times y+r = (q+1) \times y + r - y = q^{'} \times y^{'} + r^{'} $. Therefore, $P$ is still true. So $P$ is invariant.
  Thus, when this loop terminates, $P$ is true, which means $x=q \times y+r$, and $r \ge y$ is false when $r$ is smaller than $y$ and $r$ never minus a number larger than itself, so $0 \le r <y$.
  \end{proof}

\end{enumerate}
\end{enumerate}
%\begin{eqnarray*}
%k& = &2*a' + 3*b' +1 \\
%& = & 2*(a'-1) + 3*(b'+1)\\
%&=& 2*(a'+2) + 3*(b'-1)
%\end{eqnarray*}
%========================================================================
\end{document}